\documentclass[11pt]{article}
    \title{\textbf{Default Doc}}
    \date{\the\day/\the\month/\the\year}


\hyphenpenalty=10000
\linepenalty=-100
\binoppenalty=10000
\relpenalty=10000
\predisplaypenalty=-100

\addtolength{\oddsidemargin}{-.75in}
\addtolength{\evensidemargin}{-.75in}
\addtolength{\textwidth}{1.5in}
\addtolength{\textheight}{4cm}
\addtolength{\topmargin}{-2.5cm}

\topskip=40pt
\parskip=5pt
\parindent=0pt
\baselineskip=15pt
%\spaceskip=.3333em plus.03em minus .02em
%\xspaceskip=.5em plus.08em minus.02em
%\hbadness=10000

\usepackage{amsmath}
\usepackage{amssymb}

\usepackage[svgnames]{xcolor}
\usepackage{transparent}
\usepackage{svg}
\usepackage{svg-extract}

\newcommand{\reals}{\mathbb{R}}
\newcommand{\complex}{\mathbb{C}}
\newcommand{\nats}{\mathbb{N}}
\newcommand{\integers}{\mathbb{Z}}
\newcommand{\rationals}{\mathbb{Q}}

\newcommand{\powerset}{\mathcal{P}}

\newcommand{\inv}{{-1}}

\usepackage{tocloft}
\setlength\cftsecnumwidth{7em}
\setlength\cftsubsecnumwidth{7em}
%\setlength\cftsubsecnumwidth{10em}
%\cftsetpnumwidth{5em}
\renewcommand\cftchapafterpnum{\vspace{7pt}}
\renewcommand\cftsecafterpnum{\vspace{5pt}}
\renewcommand\cftsubsecafterpnum{\vspace{3.5pt}}

\newcommand{\CreateFirstPage}{%
\maketitle%
\thispagestyle{empty}%
\tableofcontents%
\label{TableOfContents}%
}

\counterwithout*{section}{chapter}

\makeatletter
\def\appliunisecs#1{\expandafter\@appliunisecs\csname c@#1\endcsname}
\def\@appliunisecs#1{%
  \ifcase#1\or philofmath\or proptypes\or maththink\or realnumsax\or seqlimsinR\or openlimsR\or funclimsR\or UNNAMED\or UNNAMED\or UNNAMED\or
   UNNAMED\or UNNAMED\or UNNAMED\or UNNAMED\or UNNAMED\or UNNAMED\or UNNAMED\or UNNAMED\or UNNAMED\or UNNAMED\or UNNAMED\or UNNAMED\or UNNAMED\or UNNAMED\or
    UNNAMED\or UNNAMED\else\@ctrerr\fi}
\makeatother

\renewcommand*{\thesection}{\appliunisecs{section}}
% this is for macros that need to be interpretted differently on web
% i.e. svgs have to go through a very different process on latex than
% on html
%

\usepackage{hyperref}

\hypersetup{
    colorlinks=true,
    linkcolor=black,
    filecolor=magenta,      
    urlcolor=blue,
	citecolor=black
    }

\newcommand{\figuresvgwithcaption}[2]{%
	\begin{figure}[tbh]%
		\centering%
		\includesvg{#1}%
		\caption{\centering #2}%
	\end{figure}%
}
%
\newcommand{\squote}[1]{`#1'}
\newcommand{\dquote}[1]{``#1"}

\renewcommand{\theenumi}{\alph{enumi}}
\newcounter{statements}[section]
\renewcommand{\thestatements}{\thesection.\arabic{statements}}

\usepackage{tcolorbox}
\usepackage{amsthm}

\tcbuselibrary{breakable}


\makeatletter
\newcommand{\@myifempty}[3]{\if\relax\detokenize{#1}\textnormal{#2}\else\textnormal{#3}\fi}


\newtcolorbox[use counter=statements]%
	{label definition}%
	[2]%
	[]%
	{%
		breakable,%
		colback=white,%
		colframe=LightGreen,%
		coltitle=black,%
		title=Definition \thestatements\@myifempty{#1}{}{\quad---\quad (#1)},%
		label=def:#2%
	}

\newtcolorbox[use counter=statements]%
	{definition}%
	[1]%
	[]%
	{%
		breakable,%
		colback=white,%
		colframe=LightGreen,%
		coltitle=black,%
		title=Definition \thestatements\@myifempty{#1}{}{\quad---\quad (#1)},%
	}
	
\newtcolorbox[use counter=statements]%
	{label theorem}%
	[2]%
	[]%
	{%
		breakable,%
		colback=white,%
		colframe=LightCoral,%
		coltitle=black,%
		title=Theorem \thestatements\@myifempty{#1}{}{\quad---\quad (#1)},%
		label=thm:#2%
	}

\newtcolorbox[use counter=statements]%
	{theorem}%
	[1]%
	[]
	{%
		breakable,%
		colback=white,%
		colframe=LightCoral,%
		coltitle=black,%
		title=Theorem \thestatements\@myifempty{#1}{}{\quad---\quad (#1)},%
	}
	
\newtcolorbox[use counter=statements]%
	{label proposition}%
	[2]%
	[]%
	{%
		breakable,%
		colback=white,%
		colframe=Violet,%
		coltitle=black,%
		title=Proposition \thestatements\@myifempty{#1}{}{\quad---\quad (#1)},%
		label=pro:#2%
	}

\newtcolorbox[use counter=statements]%
	{proposition}%
	[1]%
	[]
	{%
		breakable,%
		colback=white,%
		colframe=Violet,%
		coltitle=black,%
		title=Proposition \thestatements\@myifempty{#1}{}{\quad---\quad (#1)},%
	}

\newtcolorbox%
	{label proof}%
	[3]%
	[Proof.]%
	{%
		colback=white,%
		colframe=LightBlue,%
		coltitle=black,%
		title=#1,%
		label=prf:#3,%
		breakable%
	}

%\newenvironment{label proof}%
%	[3][Proof.][]%
%	{\begin{@ label proof}[#1]{#3}}%
%	{\end{@ label proof}}
	

\newtcolorbox%
	{my proof}%
	[2]%
	[Proof.]
	{%
		colback=white,%
		colframe=LightBlue,%
		coltitle=black,%
		title=#1,%
		breakable%
	}
	
\newtcolorbox[use counter=statements]%
	{label lemma}%
	[2]%
	[]%
	{%
		colback=white,%
		colframe=LightPink,%
		coltitle=black,%
		title=Lemma \thestatements\@myifempty{#1}{}{\quad---\quad (#1)},%
		label=lem:#2%
	}

\newtcolorbox[use counter=statements]%
	{lemma}%
	[1]%
	[]
	{%
		colback=white,%
		colframe=LightPink,%
		coltitle=black,%
		title=Lemma \thestatements\@myifempty{#1}{}{\quad---\quad (#1)},%
	}
	
\newtcolorbox[use counter=statements]%
	{label corollary}%
	[2]%
	[]%
	{%
		colback=white,%
		colframe=LightYellow,%
		coltitle=black,%
		title=Corollary \thestatements\@myifempty{#1}{}{\quad---\quad (#1)},%
		label=crl:#2,%
		breakable%
	}

\newtcolorbox[use counter=statements]%
	{corollary}%
	[1]%
	[]
	{%
		colback=white,%
		colframe=Khaki,%
		coltitle=black,%
		title=Corollary \thestatements\@myifempty{#1}{}{\quad---\quad (#1)},%
		breakable%
	}
	
\newtcolorbox[use counter=statements]%
	{label notation}%
	[2]%
	[]%
	{%
		colback=white,%
		colframe=LightGrey,%
		coltitle=black,%
		title=Notation \thestatements\@myifempty{#1}{}{\quad---\quad (#1)},%
		label=crl:#2%
	}

\newtcolorbox[use counter=statements]%
	{notation}%
	[1]%
	[]
	{%
		colback=white,%
		colframe=LightGrey,%
		coltitle=black,%
		title=Notation \thestatements\@myifempty{#1}{}{\quad---\quad (#1)},%
	}
	
\makeatother




\begin{document}

In this section it is pertinant to stop and examine the properties of real numbers again, as it turns out that the axiom of completeness has much more significant consequences than its mere statement might suggest. That is, we have mentioned many times in the previous section the notion of a \squote{region of convergence}, and while we have formally discussed the idea of \squote{convergence}, we have not yet formally discussed the notion of a \squote{region}.

Such a concept is deceptively complicated, as it cuts to the heart of what separates the discrete regime of numbers from continuous mathematics, and more importantly, is at the heart of why it is meaningful for us to speak of numbers as a model for geometry at all. The real numbers are constructed with the intention that they have \emph{no gaps}. In mathematics, we speak about this property positively and instead say that \dquote{the real numbers are \textbf{complete}}. There are multiple criteria for completeness (which in certain hairsplitting settings even disagree quite badly) but for the purposes of real analysis we are generally concerned with \emph{Cauchy completeness}.

In fact Cauchy completeness is exactly the property that in a space, all Cauchy sequences converge, a property that we will return to in many settings since it is profoundly non-trivial. Consider for instance the set of rational numbers, the numbers formed from fractions of integers, $\rationals$. Obviously numbers such as $\sqrt{2}$ are not included in such a set, and yet it is easy to construct a sequence of rational numbers that converges to an irrational. In fact it is relatively easy using the reasoning of the \hyperref[thm:MonotoneConvergenceTheoremRSeq]{monotone convergence theorem} to simply imagine a sequence of only rational numbers which is monotonic increasing but bounded above by $\sqrt{2}$ (since irrational numbers and rational numbers may be placed on the same ordered axis) which is at every point rational and yet always approaching an irrational number. It is possible to speak of such a sequence converging and yet it must converge to something outside of the space it is defined in, that of the rationals.

It is precisely because we have the axiom of completeness that we may point to a subset of the rational numbers, bounded above by number which is not itself rational and thus a \squote{gap} as far as the rationals are concerned, and fill that gap with whatever needs to be there to \emph{complete} the space with a supremum of the set. 

\end{document}