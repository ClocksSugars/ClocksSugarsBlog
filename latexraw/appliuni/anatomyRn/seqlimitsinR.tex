\documentclass[11pt]{article}
    \title{\textbf{Default Doc}}
    \date{\the\day/\the\month/\the\year}


\hyphenpenalty=10000
\linepenalty=-100
\binoppenalty=10000
\relpenalty=10000
\predisplaypenalty=-100

\addtolength{\oddsidemargin}{-.75in}
\addtolength{\evensidemargin}{-.75in}
\addtolength{\textwidth}{1.5in}
\addtolength{\textheight}{4cm}
\addtolength{\topmargin}{-2.5cm}

\topskip=40pt
\parskip=5pt
\parindent=0pt
\baselineskip=15pt
%\spaceskip=.3333em plus.03em minus .02em
%\xspaceskip=.5em plus.08em minus.02em
%\hbadness=10000

\usepackage{amsmath}
\usepackage{amssymb}

\usepackage[svgnames]{xcolor}
\usepackage{transparent}
\usepackage{svg}
\usepackage{svg-extract}

\newcommand{\reals}{\mathbb{R}}
\newcommand{\complex}{\mathbb{C}}
\newcommand{\nats}{\mathbb{N}}
\newcommand{\integers}{\mathbb{Z}}
\newcommand{\rationals}{\mathbb{Q}}

\newcommand{\powerset}{\mathcal{P}}

\newcommand{\inv}{{-1}}

\usepackage{tocloft}
\setlength\cftsecnumwidth{7em}
\setlength\cftsubsecnumwidth{7em}
%\setlength\cftsubsecnumwidth{10em}
%\cftsetpnumwidth{5em}
\renewcommand\cftchapafterpnum{\vspace{7pt}}
\renewcommand\cftsecafterpnum{\vspace{5pt}}
\renewcommand\cftsubsecafterpnum{\vspace{3.5pt}}

\newcommand{\CreateFirstPage}{%
\maketitle%
\thispagestyle{empty}%
\tableofcontents%
\label{TableOfContents}%
}

\counterwithout*{section}{chapter}

\makeatletter
\def\appliunisecs#1{\expandafter\@appliunisecs\csname c@#1\endcsname}
\def\@appliunisecs#1{%
  \ifcase#1\or philofmath\or proptypes\or maththink\or realnumsax\or seqlimsinR\or openlimsR\or funclimsR\or UNNAMED\or UNNAMED\or UNNAMED\or
   UNNAMED\or UNNAMED\or UNNAMED\or UNNAMED\or UNNAMED\or UNNAMED\or UNNAMED\or UNNAMED\or UNNAMED\or UNNAMED\or UNNAMED\or UNNAMED\or UNNAMED\or UNNAMED\or
    UNNAMED\or UNNAMED\else\@ctrerr\fi}
\makeatother

\renewcommand*{\thesection}{\appliunisecs{section}}
% this is for macros that need to be interpretted differently on web
% i.e. svgs have to go through a very different process on latex than
% on html
%

\usepackage{hyperref}

\hypersetup{
    colorlinks=true,
    linkcolor=black,
    filecolor=magenta,      
    urlcolor=blue,
	citecolor=black
    }

\newcommand{\figuresvgwithcaption}[2]{%
	\begin{figure}[tbh]%
		\centering%
		\includesvg{#1}%
		\caption{\centering #2}%
	\end{figure}%
}
%
\newcommand{\squote}[1]{`#1'}
\newcommand{\dquote}[1]{``#1"}

\renewcommand{\theenumi}{\alph{enumi}}
\newcounter{statements}[section]
\renewcommand{\thestatements}{\thesection.\arabic{statements}}

\usepackage{tcolorbox}
\usepackage{amsthm}

\tcbuselibrary{breakable}


\makeatletter
\newcommand{\@myifempty}[3]{\if\relax\detokenize{#1}\textnormal{#2}\else\textnormal{#3}\fi}


\newtcolorbox[use counter=statements]%
	{label definition}%
	[2]%
	[]%
	{%
		breakable,%
		colback=white,%
		colframe=LightGreen,%
		coltitle=black,%
		title=Definition \thestatements\@myifempty{#1}{}{\quad---\quad (#1)},%
		label=def:#2%
	}

\newtcolorbox[use counter=statements]%
	{definition}%
	[1]%
	[]%
	{%
		breakable,%
		colback=white,%
		colframe=LightGreen,%
		coltitle=black,%
		title=Definition \thestatements\@myifempty{#1}{}{\quad---\quad (#1)},%
	}
	
\newtcolorbox[use counter=statements]%
	{label theorem}%
	[2]%
	[]%
	{%
		breakable,%
		colback=white,%
		colframe=LightCoral,%
		coltitle=black,%
		title=Theorem \thestatements\@myifempty{#1}{}{\quad---\quad (#1)},%
		label=thm:#2%
	}

\newtcolorbox[use counter=statements]%
	{theorem}%
	[1]%
	[]
	{%
		breakable,%
		colback=white,%
		colframe=LightCoral,%
		coltitle=black,%
		title=Theorem \thestatements\@myifempty{#1}{}{\quad---\quad (#1)},%
	}
	
\newtcolorbox[use counter=statements]%
	{label proposition}%
	[2]%
	[]%
	{%
		breakable,%
		colback=white,%
		colframe=Violet,%
		coltitle=black,%
		title=Proposition \thestatements\@myifempty{#1}{}{\quad---\quad (#1)},%
		label=pro:#2%
	}

\newtcolorbox[use counter=statements]%
	{proposition}%
	[1]%
	[]
	{%
		breakable,%
		colback=white,%
		colframe=Violet,%
		coltitle=black,%
		title=Proposition \thestatements\@myifempty{#1}{}{\quad---\quad (#1)},%
	}

\newtcolorbox%
	{label proof}%
	[3]%
	[Proof.]%
	{%
		colback=white,%
		colframe=LightBlue,%
		coltitle=black,%
		title=#1,%
		label=prf:#3,%
		breakable%
	}

%\newenvironment{label proof}%
%	[3][Proof.][]%
%	{\begin{@ label proof}[#1]{#3}}%
%	{\end{@ label proof}}
	

\newtcolorbox%
	{my proof}%
	[2]%
	[Proof.]
	{%
		colback=white,%
		colframe=LightBlue,%
		coltitle=black,%
		title=#1,%
		breakable%
	}
	
\newtcolorbox[use counter=statements]%
	{label lemma}%
	[2]%
	[]%
	{%
		colback=white,%
		colframe=LightPink,%
		coltitle=black,%
		title=Lemma \thestatements\@myifempty{#1}{}{\quad---\quad (#1)},%
		label=lem:#2%
	}

\newtcolorbox[use counter=statements]%
	{lemma}%
	[1]%
	[]
	{%
		colback=white,%
		colframe=LightPink,%
		coltitle=black,%
		title=Lemma \thestatements\@myifempty{#1}{}{\quad---\quad (#1)},%
	}
	
\newtcolorbox[use counter=statements]%
	{label corollary}%
	[2]%
	[]%
	{%
		colback=white,%
		colframe=LightYellow,%
		coltitle=black,%
		title=Corollary \thestatements\@myifempty{#1}{}{\quad---\quad (#1)},%
		label=crl:#2,%
		breakable%
	}

\newtcolorbox[use counter=statements]%
	{corollary}%
	[1]%
	[]
	{%
		colback=white,%
		colframe=Khaki,%
		coltitle=black,%
		title=Corollary \thestatements\@myifempty{#1}{}{\quad---\quad (#1)},%
		breakable%
	}
	
\newtcolorbox[use counter=statements]%
	{label notation}%
	[2]%
	[]%
	{%
		colback=white,%
		colframe=LightGrey,%
		coltitle=black,%
		title=Notation \thestatements\@myifempty{#1}{}{\quad---\quad (#1)},%
		label=crl:#2%
	}

\newtcolorbox[use counter=statements]%
	{notation}%
	[1]%
	[]
	{%
		colback=white,%
		colframe=LightGrey,%
		coltitle=black,%
		title=Notation \thestatements\@myifempty{#1}{}{\quad---\quad (#1)},%
	}
	
\makeatother




\begin{document}

Here we can begin to substantiate at least one notion of what happens \emph{at infinity}, in so far as you could ever say that, which, for our formal purposes, I will continue to say we cannot. In fact, one could execute a proof of the Archimedian property from the previous section in the opposite manner, showing not that there exist no infinitessimals in the reals but no infinity objects in $\reals$. So infinity believers are in a rough spot right now. We certainly know what we mean informally when we want to discuss infinity, but we cannot actually do so without producing contradictions via the Archimedean property. It is perhaps one of the motivations of real analysis that we are able to construct ways to launder infinity into our work.

Before we are ready to use infinity to speak of things infinitessimally close to a point however, it will be easier to develop our conceptions and mathematical tools if we focus on the simplest kind of infinity first, not an infinity we imagine in $\reals$ (or indeed an infinitessimal), but an infinity in $\nats$. This is no mere building block either; we will continue to use limits of countable sequences for probably the entirety of this text since they provide an alternate, often simpler, way to prove things or make statements.

Our notion of \squote{what is at infinity} will be defined not as a literal infinity but as what happens \squote{on the way there}; consequently it will be easier to speak about \squote{on the way there} if the process of \squote{going there} is a countable one, i.e. we will work with sequences instead of paths.

\begin{label definition}[Sequence on $\reals$]{NSequenceOnR}
A sequence is an ordered set of real numbers indexed by $\nats$, or equivalently, it is a function $a: \nats \to \reals$ which we write $(a_n)_{n \in \nats}$ and denote elements of the ordered set (or equivalently outputs of the function) as $a_n$ for the corresponding counting number $n$.

Sometimes (and in some math texts) we use $a_n$ with $n$ completely unspecified as a shorthand to refer to the whole sequence $(a_n)_{n\in \nats}$.
\end{label definition}

\begin{label definition}[Limits on Sequences]{SequenceLimit}
We say that a sequence $(a_n)_{n \in \nats}$ \textbf{converges} to $a \in \reals$ or \squote{has \textbf{limit}} $a \in \reals$ if the following is true: 

for all $\varepsilon > 0$, there exists $N \in \nats$ such that for all $n \ge N$, we have $|a_n - a| \le \varepsilon$.

This condition is often formally written: \begin{gather*}
\forall \varepsilon > 0, \hspace{0.1em} \exists N \in \nats, \hspace{0.1em} (n \ge N) \Rightarrow \big(|a_n - a| \le \varepsilon \big).
\end{gather*}

When the sequence converges to $a$, we write \begin{gather*}
\lim_{n \to \infty} a_n = a
\end{gather*}
or say that $a_n \to a$ as $n \to \infty$. (The symbol $\varepsilon$ is the greek letter \squote{epsilon}.)
\end{label definition}

Personally I found the statement for the requirement of a limit quite difficult to parse the first time I saw it, not having been practiced at proposition heavy mathematics. And especially I was frustrated that I did not see why this seemingly arbitrary requirement had anything to do with infinity. I hope that the following description will mean you share no such grievance.

First, the limit requirement. The way this is to be read is thus. Imagine I gave you a sheet of metal and I told you, \dquote{hey, I found this sheet of metal, and it's \emph{perfectly smooth}}. You're suspicious, so you run your finger over it and feel that it is smooth, meaning that it appears perfectly smooth at least to the detection standard of your fingers, which are not very precise instruments. So you're still suspicious, and you get a magnifying glass and look it over for imperfections, and you still find none. Next, you get a microscope, and under the microscope, the sheet of metal still appears to be perfectly smooth. It is not that you need to view the atomic structure of the sheet to know that it is perfectly smooth, but in particular that you know it is perfectly smooth once you can \emph{trust} that \dquote{no matter how closely I look, my measuring device will always tell me the sheet is smooth}. This is our definition of \squote{perfect} here; that the property remains true no matter how much more precision you use to check if the property holds.

The definition of the limit is exactly the same as above, but instead of perfect smoothness we are saying that the sequence \emph{becomes perfectly close} to $a$.

I tell you that $(a_n)_{n \in \nats}$ converges to $a$. You're suspicious, so you pick some measuring threshhold $\varepsilon \in \reals$ which is $\varepsilon > 0$, and you ask me if I can find some point in the sequence where the entire sequence after that point is less than $\varepsilon$ away from $a$. That is, you ask me to find some $N \in \nats$ such that for all $n \ge N$, we satisfy $|a_n - a| \le \varepsilon$; the condition $n \ge N$ ensures that $a_n$ and all $a_n$ after it is past the point of sufficient closeness. You're still suspicious, so you pick a smaller $\varepsilon$, and once again I am able to find a larger $N \in \nats$ for which all $a_n$ with $n$ after $N$ are less than $\varepsilon$ away from $a$. You pick an even smaller $\varepsilon$, but this time, instead of giving you a \emph{particular} $N\in \nats$, I give you a method by which you can find your own $N \in \nats$ for any $\varepsilon$. By providing this method, you can trust that no matter how small a $\varepsilon$, there will exist a satisfactory $N$, and you can \emph{trust} that the sequence converges. This is what it means to prove the limit.

\begin{example}[$1/n^2$ goes to $0$ as $n$ goes to $\infty$]
Let us show that the sequence $\left(\frac{1}{n^2} \right)_{n \in \nats}$ converges to $0$. Observe first that $\left|\frac{1}{n^2} - 0\right| = \frac{1}{n^2}$ since $\frac{1}{n^2}$ is always positive. Moreover, the function (and the sequence it defines) $n \mapsto \frac{1}{n^2}$ is \emph{monotonically decreasing} as defined in the previous section, meaning that for all $n$, $\frac1{n^2} > \frac1{(n+1)^2}$. (If the reason why in this case is not obvious to you, set $n = 2$ and remember that a quarter of a pizza is larger than a ninth of one. See corollaries at the end of the previous chapter). This means that if we pick $\varepsilon = 1/N^2$, this would satisfy the limit $|1/n^2 - 0| \le 1/N^2$ for all $n \ge N$ since when $n$ is greater than $N$, by the monotonically decreasing property, $1/n^2$ must be smaller than $1/N^2$.

But the whole point of the reasoning of the limit is that we, as the ones proving the limit, cannot pick $\varepsilon$. We must imagine someone else picking $\varepsilon$ and ourselves making a procedure to find $N$ for which $n \ge N$ implies $|1/n^2 - 0| < \varepsilon$. Still, noticing that the limit is satisfied when $\varepsilon = 1/N^2$ helps us, since we can invert this relation. Naively we can take \begin{align*}
\varepsilon &= \frac{1}{N^2} \\
\frac{1}{\varepsilon} &= N^2 \\
\frac{1}{\sqrt{\varepsilon}} &= N
\end{align*}
And if this alone worked, then in fact we would be done. Setting $N$ like this would indeed mean that when $n \ge N$ then we have $|1/n^2 - 0| < \varepsilon$. The only trouble is that $N \in \nats$ needs to be a counting number, and we have no guarentee that the $\varepsilon$ picked will produce a counting number when we take its inverse squareroot. This is easily solved however by rounding up $1/\sqrt{\varepsilon}$ to the largest whole number, an operation called taking the \emph{ceiling} and written \begin{gather*}
N = \left\lceil \frac{1}{\sqrt{\varepsilon}} \right\rceil \ge \frac{1}{\sqrt{\varepsilon}}.
\end{gather*}
with the property that it is greater than or equal to its counterpart that is not rounded up. Now when we are given some $\varepsilon$ we are able to automatically produce some $N$ for which $n \ge N$ will satisfy $|1/n^2 - 0| \le \varepsilon$.

Let's test this to be certain. Say we are given $\varepsilon$ and thus set $N = \lceil 1/\sqrt{\varepsilon} \rceil$. Since $N$ is rounded up, first and foremost we have $N \ge 1/\sqrt{\varepsilon}$ larger than its unrounded version.

Now consider that since $x \mapsto 1/x$ is a monotonic decreasing function on positive numbers (which $\varepsilon$ is required to be), we deduce from $N \ge 1/\sqrt{\varepsilon}$ that \begin{gather*}
\frac{1}{N} \le \sqrt{\varepsilon}
\end{gather*}
flipping the ordering by the monotonic decreasing property in \ref{def:MonotonicFunction}. We also have that $x \mapsto x^2$ is a monotonic increasing function on positive numbers, so undoing the square root will preserve the ordering. \begin{gather*}
\frac{1}{N^2} \le \varepsilon.
\end{gather*}
Recall that the property we wanted out of $N$ was that $|1/n^2 - 0| \le \varepsilon$ was only true when $n \ge N$, so let us operate on that next. Once again, $x \mapsto 1/x$ will flip the sign, and $x \mapsto x^2$ will preserve it. \begin{gather*}
n \ge N \\
\frac{1}{n} \le \frac{1}{N} \\
\frac{1}{n^2} \le \frac{1}{N^2}
\end{gather*}
We also had that $|1/n^2 - 0| = 1/n^2$. So together we have a transitivity between equality and addition, \begin{gather*}
\left| \frac{1}{n^2} - 0\right| = \frac{1}{n^2} \le \frac{1}{N^2} \le \epsilon
\end{gather*}
implying \begin{gather*}
\left| \frac{1}{n^2} - 0\right| \le \epsilon
\end{gather*}
as we had wanted. Notice that we proved this, as we needed to, \emph{for any} $\varepsilon > 0$ that \emph{there exists some} $N \in \nats$ which for us was $N = \lceil 1/\sqrt{\varepsilon} \rceil$, and if we assumed $n \ge N$ then we could deduce $|1/n^2 - 0| \le \varepsilon$. This is the pattern of a proof of limit.
\end{example}

In general, the pattern of proving a limit is roughly as above; in particular, we first identify how small a $\varepsilon$ a given $N$ can satisfy, which we did when we noticed that $\varepsilon = 1/N^2$ will mean we are trying to prove $n \ge N$ implies $1/n^2 \le 1/N^2$, which we know to be true immediately from the monotonic decreasing property of $x \mapsto 1/x^2$. Next, we treat the earlier relationship $\varepsilon = 1/N^2$ as a function $\varepsilon \mapsto 1/N^2$ and we see if we can invert it, finding either some way to calculate $1/N^2 \mapsto \varepsilon$ or the next best thing; this ended up being our $N \mapsto 1/\sqrt{\varepsilon}$ as our inverse function, and we had to round it up to make it a whole number. Most proofs will omit these steps, since the crux of the proof is not that we can find some relationship but rather that a relationship exists and that it \emph{works}. So having found our relationship $N = \lceil 1/\sqrt{\varepsilon} \rceil$ we try to in some sense unwrap whatever transformations we've made to the function while taking $n$ along for the journey; that is, we map $n \ge N$ to $1/n \le 1/N$ together, then square them together. At this point, some minor amount of symbolic massaging will generally get you your limit.

We could do more examples (and indeed I may have to return here and add more) but the majority of the limits we will prove over this chapter are limits related to other limits. So instead of the particular technique described above, what we will frequently be doing is saying that some sequence $(a_n)_{n \in \nats}$ converges to $a$, and that since it does, we can take the existence of the $N$ which it implies and use it to imply that some other sequence $(b_n)_{n\in \nats}$ also converges to some $b$.

At the beginning of chapter 1 we discussed the idea of the development of a theory; this involved defining core concepts, sub-concepts, relationships between those concepts, and finally substantiation of the implication of the theory onto other areas of study. We have now defined our core concepts for our theory of limits, and so in line with stage two, we must define sub-concepts and properties.

\begin{label definition}[Subsequences on $\mathbb{R}$]{NSubsequenceOnR}
Let $(a_n)_{n\in \nats}$ and $(b_n)_{n\in \nats}$ be sequences, and let there exist a monotonic increasing map $\varphi\colon \nats \to \nats$ (so for all $n > m$ we have $\varphi(n) > \varphi(m)$). If $b_n = a_{\varphi(n)}$ for all $n \in \nats$ then we call $(b_n)_{n\in\nats}$ a \textbf{subsequence} of $(a_n)_{n\in \nats}$. That is, a subsequence is a sequence that skips elements, or that is missing elements.
\end{label definition}

A subsequence can do a few things for us. We can define a subsequence with $\varphi\colon n \mapsto n + m$ where $m \in \nats$ is some number that skips the sequence forwards by a constant. We could have $\varphi\colon n \mapsto 2n$ which would skip every second entry; this is useful in cases like the sequence $\big((-1)^n\big)_{n\in \nats}$ which alternates between $1$ and $-1$ and does not converge, unless of course we were to skip all the $-1$ entries as the subsequence $\big((-1)^{\varphi(n)}\big)_{n \in \nats}$ allows us to do, making it a constant sequence.

We are now at the point where we can make a firm statement about what is and is not possible given the definitions we have constructed. If a sequence converges, no subsequence of that sequence can converge to a different value. This should make sense to us, since if we think of a converging sequence as one that becomes as close as desired to a point it converges at, then of course skipping some of the points in the sequence should simply make the subsequence converge faster.

\begin{label proposition}[Subsequences Preserve Limits]{SubsequencePreservesLimits}
Let $(a_n)_{n\in \nats}$ be a sequence which converges at $a \in \reals$. Then all subsequences $(b_n)_{n\in \nats}$ converge to $a$ as well.
\end{label proposition}

\begin{my proof}{mo}
Since we know $(a_n)_{n\in \nats}$ converges to $a$, we have that any $\varepsilon_a > 0$ we choose, there exists $N_a$ such that for all $n \ge N_a$, $|a_n - a| \le \varepsilon_a$. From this, we want to prove that for all $\varepsilon_b$, there exists $N_b$ such that for all $n \ge N_b$ we have $|b_n - a| \le \varepsilon_b$. We also know that $(b_n)_{n\in \nats}$ is a subsequence, and thus that there exists a monotonic increasing map $\varphi\colon \nats \to \nats$ for which $b_n = a_{\varphi(n)}$, and so we can rewrite the goal of our proof instead as $\varphi(n) \ge N_b$ implies $|a_{\varphi(n)} - a| \le \varepsilon_b$.

Then when we are given some $\varepsilon_b > 0$, set our $\varepsilon_a$ from earlier as $\varepsilon_a = \varepsilon_b$ so that it tells us there exists a $N_a$ for which $n \ge N_a$ implies $|a_n - a| \le \varepsilon_a = \varepsilon_b$. Because $\varphi$ is an increasing map, it has $n \le \varphi(n)$, which is the same as either $n = \varphi(n)$ or $n < \varphi(n)$. If it is the former, then the statement $\varphi(n) \ge N_b$ implies $|a_{\varphi(n)} - a| \le \varepsilon_b$ is literally $n \ge N_b$ implies $|a_n - a| \le \varepsilon_b = \varepsilon_a$, and so we can set $N_b = N_a$ and \emph{inherit} the limit directly from $(a_n)_{n\in \nats}$.

In the other case where $\varphi(n) > n$, notice that if we knew $n \ge N_a$ then we would have by transitivity that \begin{gather*}
\varphi(n) > n \ge N_a
\end{gather*}
meaning that we have $\varphi(n) \ge N_a$. But since we know $(a_m)_{m\in \nats}$ converges, we have that $m \ge N_a$ implies $|a_m - a| \le \varepsilon_a$, and this holds for all $m$ (which we write instead of $n$ to avoid confusion about which implications we are using) including if we write $\varphi(n)$ in place of $m$. So, since we know $\varphi(n) \ge N_a$, we know $|a_{\varphi(n)} - a| \le \varepsilon_a$. Earlier we said though that $b_n = a_{\varphi(n)}$, and we also set $\varepsilon_a =\varepsilon_b$. This means that we have shown $n \ge N_a$, together with other facts we know to be true, implies \begin{gather*}
|b_n - a| = |a_{\varphi(n)} - a| \le \varepsilon_a \le \varepsilon_b \\
|b_n - a| \le \varepsilon_b
\end{gather*}
So we need a way to know that $n \ge N_a$, and remember that our main goal is to show $n \ge N_b$ implies $|b_n - a| \le \varepsilon_b$, and we have not yet decided what $N_b$ should be. We can solve this conundrum by setting $N_b = N_a$, the same $N_a$ we were told exists since $(a_n)_{n\in \nats}$ converges when we set $\varepsilon_a = \varepsilon_b$. This means that assuming $n \ge N_b$ implies $n \ge N_a$, which we showed earlier implies that $|b_n - a| \le \varepsilon_b$. So we are done.
\end{my proof}

We can forecast some intuitions if we think about the above statement; recall earlier that we are studying sequences in place of paths, but for the sake of intuition, imagine a path, a line through $\reals$ or a dot moving around on a journey in $\reals$ and tracing its path. A converging sequence can be thought of as the dot moving around in $\reals$ until it eventually settles into a region around the point it will eventually converge to; indeed it must settle in such a fashion, or else it would be impossible to say that eventually the rest of the sequence past some $N$ will be less than $\varepsilon$ distance away. Then of course, we reason that skipping the start of this journey or speeding this process along should only make it settle faster. In this way, the above proposition is obvious.

However we can draw more insights from this picture, each which will translate into theorems.

Now imagine the path it takes; that region it settles into must be bounded, because just as eventually (for some $N\in \nats$) it will be less than $\varepsilon$ away from its destination, its limit, it is bounded on either side of the limit by a distance of $\varepsilon$. Prior to this (i.e. for $n < N$), it is not bounded, except that on any \emph{particular} sequence it takes \emph{particular} values $a_n$ for each $n < N$, and let us not forget that this moving dot moves not continuously but from each $a_n$ to the next $a_{n+1}$. As such, although it will have an infinite amount of time to move \emph{closer} to its limit, it has only a finite amount of time (since $0 < n < N$) to get within $\varepsilon$ of its limit, and only a finite number of stops $a_1, a_2, ...$ before it does so. And as a finite number of particular points, we may simply draw a circle around them and bound them as well. So we conclude that any sequence that converges will have the entire path it takes bounded as well.

Conversely, if a sequence is bounded but not necessarily converging, then it has an infinite number of points $a_n$ to stop at and only a finite amount of space to place them. Those points have to go somewhere, and by skipping the points that do not move closer to some infinite collection of points, we can find a subsequence which converges even if the main sequence itself does not.

By similar reasoning, we can also say that a sequence which has an upper bound but is monotonic increasing must keep going up yet only has a finite amount of up it is allowed to go; it may not necessarily stop somewhere, but it must either stop or make its movements upwards slow down until the manner of its movements look like a convergence, and so the upper-bounded monotonic increasing sequence must too converge.

There is one more extremely valuable insight we can glean from this kind of pictoral thinking. There is a sort of parallel criterion to the limit, another thing we could prove which is ultimately the same as proving convergence which may be easier to prove sometimes. That is, the same way we picture a sequence jumping from $a_n$ to $a_{n+1}$ until it is bounded in a region around its convergence, it must also be bounded around its own path. If a sequence were to converge, one should hope that the steps it takes should get smaller and smaller over time, and so we can imagine a converging sequence as not just being $\varepsilon$ away from its limit, but perhaps $\varepsilon$ away from its next point and all points after that.

This notion of a \squote{region of convergence} is not merely imaginary, but to substantiate it, we will need to learn more strange things about the real numbers, which we will do in the next section. Once we do, we will find that it is a concept of utmost importance.

Now let us translate these intuitive deductions into formal ones.

\begin{label corollary}[Subsequence Limit Theorem]{SubsequenceLimitTheorem}
Let $(a_n)$ be a sequence and $(b_n)_{n\in \nats}$ be a subsequence which converges to $b\in \reals$. Then if $(a_n)_{n \in \nats}$ to $a \in \reals$, we have $a = b$.
\end{label corollary}

This follows directly from the previous proposition, and is in some sense a direct restatement of subsequences preserving limits. In fact it is somewhat difficult to prove this explicitly since in order to propose a fault from which to proceed with proof by contradiction, one must begin with something that is so blatantly false on its face that the rest of the proof by contradiction is not particularly necessary.

\begin{label proposition}[Convergent Sequences Are Bounded]{BoundednessOfConvergentSequences}
If a sequence converges, then it is bounded. That is, if $(a_n)_{n\in \nats}$ is a sequence which converges to $a$, then there exists some $b_- \in \reals$ and some $b_+ \in \reals$ which satisfy \begin{gather*}
b_- \le a_n \le b_+
\end{gather*}
for all $n\in \nats$.
\end{label proposition}

\begin{my proof}{m}

We formalize almost exactly the story we told above. For some $\varepsilon > 0$ we have that there exists $N \in \nats$ such that when $n \ge N$, $|a_n - a| \le \varepsilon$, since by hypothesis $(a_n)_{n\in \nats}$ converges. Then fix some choice of $\varepsilon > 0$ (although the particular choice does not matter) so that it implies the existence of an appropriate $N \in \nats$ with the properties we expect for convergence. We'll set a variable $\hat{b}_+ = \sup\{|a_n - a| \mid n \ge N \}$ to be the largest distance, $|a_n - a|$, the sequence takes from its limit when $n \ge N$; this means $a + \hat{b}$ will be greater than all $a_n$ after $n \ge N$.

Then $a+\hat{b}$ would be an upper bound for $(a_n)_{n\in \nats}$ if not for the fact that some $a_n$ with $n < N$ might be greater than this upper bound; to fix this, we will define an upper bound for all $a_n$ before $n < N$, and we do this first by defining the set of points before $n < N$, $A_N = \{a_n \mid n < N\}$ (strictly this is not a set comprehension but an image of $\{n \in \nats \mid n < N\}$ through the sequence acting as a function $a_\square \colon \nats \to \reals$). With this set, we can take its maximum $\max(A_n)$ and either $\max(A_N)$ or $a+\hat{b}$ will be the upper bound for the entire sequence, so let us define \begin{gather*}
b_+ = \max \big\{ \max(A_N), a+ \hat{b}_+ \big\}
\end{gather*}

Since $A_N$ is literally contains $N - 1$ or less numbers, this quantity can be computed so long as we have a method to find $N$ from $\varepsilon$, which is guaranteed by hypothesis of convergence. The corresponding proof to calculate $b_-$, the lower bound, is exactly the same but with minimums instead of maximums and $a - \hat{b}$. \begin{gather*}
b_- = \min \big\{ \min(A_N), a - \hat{b} \big\}
\end{gather*}
\end{my proof}

\begin{label theorem}[Monotone Convergence Theorem]{MonotoneConvergenceTheoremRSeq}
Let $(a_n)_{n\in \nats}$ be a sequence which is bounded above and monotone increasing, i.e. $a_n \le a_{m}$ for all $m,n  \in \nats$ satisfying $m \ge n$, and there exists $b\in \reals$ for which $a_n \le b$ for all $n \in \nats$. Then $(a_n)_{n\in \nats}$ must converge.
\end{label theorem}

\begin{my proof}{m}
Once again, we take the story we told and formalize it. Let $A = \{ a_n \mid n\in \nats \}$ be the unordered set of all $a_n$ in the sequence. Since the set is bounded and non-empty, it has a supremum by the axiom of completeness, and we call this $s = \sup(A)$.

Now we aim to show that $s$ is the limit of $(a_n)_{n \in \nats}$. For any $\varepsilon > 0$ we are given, there must exist some $a^* \in A$ which satisfies $s - \varepsilon \le a^* \le s$, since if no such $a^* \in A$ existed in this range then the least upper bound would be $s - \varepsilon$ not $s$. Now, since $A$ is the set of all $a_n$ in the sequence, $a^* \in A$ means there must exist some $N \in \nats$ for which $a^* = a_N$, where $N$ is the $n$ which indexes the particular $a^*$ we pulled out of the sequence. Then since $(a_n)_{n\in \nats}$ is monotonic increasing, we have for all $n \ge N$ that \begin{gather*}
s - \varepsilon \le a_N \le a_n \\
s - \varepsilon \le a_n
\end{gather*}
by transitivity of ordering. Now, if we add $\varepsilon - a_n$ to both sides of this inequality, we obtain \begin{gather*}
s - \varepsilon + (\varepsilon - a_n) \le a_n + (\varepsilon - a_n) \\
s - a_n \le \varepsilon.
\end{gather*}
This is neat, because $s-a_n$ is ensured to be positive ($s$ is the \emph{supremum} of the set of all points in the sequence, and thus is larger than all $a_n$) meaning $s - a_n = |s - a_n|$. Moreover, $|s-a_n| = |a_n - s|$ because the absolute value will undo any multiplication by $-1$ inside its argument. This means we get to write \begin{gather*}
|a_n - s| \le \varepsilon
\end{gather*}
and notice that this followed as an implication from $n \ge N$ (since we could not have written $s- \varepsilon \le a_n$ otherwise). Then we have proven the limit. 
\end{my proof}

If you're interested, you should note something about the above proof. That is, unlike in previous examples, we did not actually find a method to produce $N$ for a given $\varepsilon > 0$. What we did instead was show that \emph{some} $N$ exists, and indeed if you had a picture of a number line with all the points $a_n$ marked and numbered, you would certainly be able to look between $s-\varepsilon$ and $s$ and select some $a_n$ to become our $a_N$. But instead, we selected our $a_N$ ultimately using the axiom of choice.

\begin{label definition}[Cauchy Criterion]{CauchyCriterionRSeq}
A sequence satisfies the \textbf{Cauchy criterion} if it satisfies the following property: 

for all $\varepsilon > 0$, there exists $N \in \nats$ such that for all $m,n \ge N$, we have $|a_m - a_n| \le \varepsilon$.

When this is true of a sequence, we call it a \textbf{Cauchy sequence}.
\end{label definition}

It is worth noting something about the Cauchy criterion, which is that it is not exactly a formal statement that ‘the steps the sequence takes get smaller over time’ as I said earlier. Instead it is that \emph{any} two points in the sequence have shorter distances over time, since indeed \emph{all} $m,n \ge N$ will satisfy $|a_m - a_n| \le \varepsilon$. Or one could read this as the statement that the sequence gets closer to the rest of itself over time.

This criterion is in some sense parallel to our existing definition of limit convergence; at this stage we have not proven that the Cauchy criterion has anything formally to do with limits, and yet it has a very obviously similar propositional form. Given the stories we told earlier, we can intuitively conclude some more things which we should formalize. For instance, if the sequence converges \squote{to itself}, then for similar reasons as discussed earlier for convergent sequences, it should also be bounded. And as we discussed above, a sequence which does converge should then be Cauchy.

\begin{label proposition}{CauchySequenceProperties}
\begin{enumerate}
\item Let $(a_n)_{n\in \nats}$ be a Cauchy sequence. Then the sequence is bounded.
\item Let $(a_n)_{n\in \nats}$ be a sequence which converges to $a \in \reals$. Then the sequence is Cauchy.
\end{enumerate}
\end{label proposition}

\begin{my proof}{m}

\begin{enumerate}
\item Since the sequence is assumed Cauchy, we may pick any $\varepsilon > 0$ (the choice does not matter) and the Cauchy assumption says that there must exist some $N \in \nats$ for which all $m,n \ge N$ has $|a_m - a_n| \le \varepsilon$. Now fix a choice of $m$ (again the particular choice does not matter). The proof will now proceed very similarly to the proof of proposition \ref{pro:BoundednessOfConvergentSequences}.

Let \begin{gather*}
\hat{b} = \sup\{|a_m - a_n| \mid n \ge N\}
\end{gather*}
be the supremum difference $a_m$ from all other elements of the sequence $a_n$ with $n \ge N$ (i.e. the value which bounds all other differences, the largest and then some perhaps). Then $a_m + \hat{b}$ forms an upper bound for the sequence of elements $n \ge N$. 

Since $n$ is a counting number in $\nats$, it is non-negative and thus any $n < N$ is also $0 < n < N$, meaning once again that the set of elements $a_n$ which come before $N$ are finite of length $N-1$ or less. Let's call this set $B = \{ a_n \mid n < N \}$ (once again, this is not technically a set comprehension but an image of one), so either $\max(B)$ or $a_m + \hat{b}$ will now bound the sequence. Our upper bound is then \begin{gather*}
b_+ = \max \big\{ \max(B), a_m + \hat{b} \big\}
\end{gather*}
and by a similar argument, our lower bound is \begin{gather*}
b_+ = \min \big\{ \min(B), a_m - \hat{b} \big\}.
\end{gather*}

\item Since the sequence is assumed to converge, choose some $\varepsilon > 0$; we aim to show that it is Cauchy. Since we have the limit $a_n \to a$ as $n \to \infty$, we may also set two epsilons $\varepsilon_m,\varepsilon_n = \varepsilon/2$, each half of our earlier mentioned epsilon. For each of these epsilon, since $(a_n)_{n\in \nats}$ converges, they induce some $N$, $N_m, N_n \in \nats$, each such that when $m \ge N_m$ or $n \ge N_n$ we have \begin{gather*}
|a_m - a| \le \varepsilon_m \\
|a_n - a| \le \varepsilon_n
\end{gather*}
respectively, where in particular $N_m = N_n$ since $\varepsilon_m = \varepsilon_n = \varepsilon/2$. Now we can apply the \hyperref[thm:RTriangleInequality]{triangle inequality}
\end{enumerate}


Return here %%%%%%%%%%%%%
\end{my proof}

\end{document}